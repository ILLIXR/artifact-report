% LaTeX template for Artifact Evaluation V20201122
%
% Prepared by 
% * Grigori Fursin (cTuning foundation, France) 2014-2020
% * Bruce Childers (University of Pittsburgh, USA) 2014
%
% See examples of this Artifact Appendix in
%  * SC'17 paper: https://dl.acm.org/citation.cfm?id=3126948
%  * CGO'17 paper: https://www.cl.cam.ac.uk/~sa614/papers/Software-Prefetching-CGO2017.pdf
%  * ACM ReQuEST-ASPLOS'18 paper: https://dl.acm.org/citation.cfm?doid=3229762.3229763
%
% (C)opyright 2014-2020
%
% CC BY 4.0 license
%

\documentclass{sigplanconf}

\usepackage[T1]{fontenc}
\usepackage{hyperref}
\usepackage{xcolor}
\usepackage{minted}
\usepackage{graphicx}

\newcommand{\todo}[1]{\textcolor{red}{TODO: #1}}
\newcommand{\zenodo}{\footnotesize \url{10.5281/zenodo.5542407}}

\begin{document}

\special{papersize=8.5in,11in}

%%%%%%%%%%%%%%%%%%%%%%%%%%%%%%%%%%%%%%%%%%%%%%%%%%%%
% When adding this appendix to your paper, 
% please remove above part
%%%%%%%%%%%%%%%%%%%%%%%%%%%%%%%%%%%%%%%%%%%%%%%%%%%%

\appendix
\section{Artifact Appendix}

\subsection{Abstract}

The artifacts consist of ILLIXR version 1, ILLIXR version 2, Monado, Godot, OpenXR test applications (AR Demo, Materials, Sponza, Platformer), analysis scripts, raw data, graphs, regression tests, and documentation.

Note that ILLIXR version 1 is the set of isolated components used for \S IVB, while ILLIXR Version 2 is the same components in an integrated system used for \S IVA. The artifact contains both. Where unspecified, we are referring to ILLIXR version 2.

We have made the effort to automate as much as possible of the installation process. For best results use a fresh install of Ubuntu 18.04 LTS or 20.04.

\subsection{Artifact check-list (meta-information)}

{\small
\begin{itemize}
  \item {\bf Program: } ILLIXR v1, ILLIXR v2, Monado, Godot, OpenXR test applications (AR Demo, Materials, Sponza, Platformer), analysis scripts.
  \item {\bf Compilation: } Make 4.2, clang 10.0, CUDA 11.1, Python 3.8 (included in install scripts)
  \item {\bf Data set: } See \S IIIC (included in artifact) and \S IIID (downloaded by program).
  \item {\bf Run-time environment: } Ubuntu 18.04 + install scripts.
  \item {\bf Hardware: } Any x86-64 system with an NVIDIA GPU can run, but one needs the hardware in \S IIIA for exact repeatability.
  \item {\bf Output: } \texttt{results/output/*} and \texttt{results/Graphs/*}.
  \item {\bf Experiments: } For each hardware platform, for each app, run ILLIXR.
  \item {\bf How much disk space required: } 5Gb, including downloaded datasets
  \item {\bf How much time is needed to prepare workflow: } 1 hour
  \item {\bf How much time is needed to complete experiments: } 20 minutes per hardware platform
  \item {\bf Publicly available: } Yes, see archive URL.
  \item {\bf Code licenses: } The system licensed under NCSA. Each component is licensed under one of: ElasticFusion License, NCSA, MIT, Simplified BSD, LGPL v3.0, LGPL v2.1, Boost Software License v1.0, GPL v3.0. See {\footnotesize \url{https://github.com/ILLIXR/ILLIXR/\#licensing-structure}}
  \item {\bf Data licenses: } Each dataset is licensed under one of: proprietary, ElasticFusion License, Creative Commons 0.
  \item {\bf Archived (provide DOI): } \zenodo
\end{itemize}
}

\subsection{Description}

\subsubsection{How to access}

One can find the version we used for this paper here (\zenodo), and a rolling release here ({\footnotesize \url{https://github.com/ILLIXR/ILLIXR}}). We suggest using the rolling release, unless exact repeatability is desired.

\subsubsection{Hardware dependencies}

One can find the hardware we used for this paper in \S IIIA. ILLIXR will still work on a generic x86-64 Ubuntu system with a GPU, but the results may not be exactly repeatable.

\subsubsection{Software dependencies}

Refer to NVIDIA's instructions for your GPU to install the proprietary NVIDIA driver and NVIDIA CUDA SDK.

{
\small
\begin{minted}{shell}
./install_deps.sh
\end{minted}
}

This script is \textit{idempotent}, so there is no harm in interrupting it and running it twice. It installs basic build tools, profiling tools, Python, ROS, conda, OpenCV 3.4, Eigen, datasets, and several other resources.

See our online documentation for more details:\newline
{\footnotesize \url{https://illixr.github.io/ILLIXR/getting_started}}

\subsubsection{Data sets}

The software pulls required datasets automatically when downloading software dependencies.

\subsection{Installation}

Not applicable.

\subsection{Experiment workflow}

First, we have to compile each application with Godot:

% TODO: automate this
{
\footnotesize
\begin{minted}{shell}
for app_path in OpenXR-Apps/*; do
  ./godot/bin/godot.x11.opt.tools.64
  # Import project (project.godot) (fig 1)
  # Export project (fig 2)
  # Select "Linux (Runnable)" (fig 3)
  # Select Custom Template="./godot/bin/godot.x11.opt.tools.64"
  # Select Export Path="./OpenXR-Apps/$app/bin"
  # Where app is replaced by $app shorname (e.g. "sponza")
done
\end{minted}
}

Then, we run each application in ILLIXR V2:

% TODO: Test this
{\scriptsize
\begin{minted}{shell}
hardware=""
# manually set to one of "jetsonlp", "jetsonhp", "desktop"
for app_path in OpenXR-Apps/*; do
  app=$(basename ${app_path})
  cmd="./ILLIXR/runner.sh ILLIXR/configs/${app}.yaml"
  ${cmd}
  nvidia-smi -q --display=UTILIZATION,POWER,TEMPERATURE \
    --loop-ms=200
  perf stat -e power/energy-cores/,power/energy-pkg/,power/energy-ram/ \
    -- ${cmd}
  mv ILLIXR/metrics results/metrics/metrics-${hardware}-${app}
done
\end{minted}
}

To switch between high- and low-power mode on Jetson

{\small
\begin{minted}{shell}
sudo jetson_clocks --restore ${lp_or_hp}_mode.txt
\end{minted}
}

\subsection{Evaluation and expected results}

Make sure to synchronize the \texttt{results/metrics/} directory from all hardware platforms onto the desktop before continuing. On the desktop, run

{\footnotesize
\begin{minted}{shell}
cd results/analysis
poetry run python3 main.py
\end{minted}
}

The output from our run is available in \texttt{metrics-snapshot} and our graphs are available in \texttt{Graphs-snapshot}.

\begin{itemize}
\item Fig 3 is {\footnotesize\texttt{results/Graphs/fps-jlp/jhp/desktop.pdf}}
\item Fig 4 is {\footnotesize\texttt{results/Graphs/timeseries-platformer-desktop-1.pdf}} and {\footnotesize\texttt{results/Graphs/timeseries-platformer-desktop-2.pdf}}
\item Fig 5 is {\footnotesize\texttt{results/Graphs/cpu-breakdown.pdf}}
\item Fig 6 is {\footnotesize\texttt{results/Graphs/power-total.pdf}} and\newline {\footnotesize\texttt{results/Graphs/power-breakdown.pdf}}
\item Fig 7 is {\footnotesize\texttt{results/Graphs/mtp-platformer.pdf}}
\item Fig 8 is {\footnotesize\texttt{results/Graphs/microarchitecture.pdf}}
\end{itemize}

To replicate Table VI, run \texttt{./ILLIXRv1/all.sh} and see the reported statistics in stdout.

To replicate Table VII, run Visual Reprojection and Hologram in NVIDIA\textsuperscript{\textcopyright} Nsight\textsuperscript{TM} Systems, and run Audio Encoding and Audio Decoding in Intel\textsuperscript{\textcopyright} VTune\textsuperscript{TM} Hotspot Analysis. See \texttt{./ILLIXRv1/all.sh} for the commands to analyze.

To replicate Fig 8, run each non-trivial command of \texttt{ILLIXRv1/all.sh} in Intel\textsuperscript{\textcopyright} VTune\textsuperscript{TM} Microarchitectural Exploration Analysis.

See \url{https://illixr.github.io/ILLIXR/legacy/v1/} for more details on running the components of ILLIXR v1.

\subsection{Experiment customization}

Here are cutomizations we anticipate:

\begin{itemize}
\item Modify or add your own app to \texttt{ILLIXR/app/*}.
\item Add your own analysis pass to \texttt{results/analysis/*.py}
\item Add your own plugin to the ILLIXR system in \texttt{ILLIXR/*}. See our online documentation for details:\newline
{\footnotesize \url{https://illixr.github.io/ILLIXR/writing_your_plugin/}}
\end{itemize}

\begin{figure}[h]
  \caption{Import a project in Godot}
  \includegraphics[width=0.7\linewidth]{import.png}
\end{figure}

\begin{figure}[h]
  \caption{Export project in Godot}
  \includegraphics[width=0.7\linewidth]{export.png}
\end{figure}

\begin{figure}[h]
  \caption{Select export options Godot}
  \includegraphics[width=0.7\linewidth]{export_options.png}
\end{figure}

%%%%%%%%%%%%%%%%%%%%%%%%%%%%%%%%%%%%%%%%%%%%%%%%%%%%
% When adding this appendix to your paper, 
% please remove below part
%%%%%%%%%%%%%%%%%%%%%%%%%%%%%%%%%%%%%%%%%%%%%%%%%%%%

\end{document}
